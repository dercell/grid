\documentclass[12pt,a4paper]{article}
\usepackage[utf8]{inputenc}
\usepackage[russian]{babel}
\usepackage{amsmath}
\usepackage{amsfonts}
\usepackage{amssymb}
\usepackage{graphicx}
\usepackage{grffile}
\usepackage{subcaption}
\usepackage{float}
\usepackage[left=2.00cm, right=2.00cm, top=2.00cm, bottom=2.00cm]{geometry}
\begin{document}

	
	\begin{titlepage}
		\begin{center}
			\large
			ПРАВИТЕЛЬСТВО РОССИЙСКОЙ ФЕДЕРАЦИИ
			
			ФЕДЕРАЛЬНОЕ ГОСУДАРСТВЕННОЕ АВТОНОМНОЕ ОБРАЗОВАТЕЛЬНОЕ 
			
			УЧРЕЖДЕНИЕ ВЫСШЕГО ОБРАЗОВАНИЯ
			
			НАЦИОНАЛЬНЫЙ ИССЛЕДОВАТЕЛЬСКИЙ УНИВЕРСИТЕТ
			
			«ВЫСШАЯ ШКОЛА ЭКОНОМИКИ»\\[12pt]
			\vspace{0.25cm}
			
			МОСКОВСКИЙ ИНСТИТУТ ЭЛЕКТРОНИКИ И МАТЕМАТИКИ
			ДЕПАРТАМЕНТ ПРИКЛАДНОЙ МАТЕМАТИКИ
			
			\vspace{2cm}
			
			{\large Отчет 
				
				о прохождении преддипломной практики}
			\bigskip
		\end{center}
		\vfill
		
		\hfill\begin{minipage}{0.35\textwidth}
			Выполнил студент 4 курса\\
			С.\,В.~Колотев\\
			
			
			Научный руководитель\\
			Л.\,Н.~Щур\\
		\end{minipage}%
		\bigskip
		\vfill
		
		
		\begin{center}
			Москва, 2016 г.
		\end{center}
	\end{titlepage}
	
	\renewcommand\contentsname{Оглавление}
	\section{Введение}
	
	\par Исследование математических и физических свойств структурных образований на плоскости, а также распознавание появления фазовых переходов вызывает большой интерес на сегодняшний день. Это объясняется двумя факторами. Первый из них - научный. Развитие стохастической эволюции Левнера(позже названной как эволюцией Шрамма-Левнера), которая связывает свойства случайного блуждания в верхней полуплоскости со свойствами конформной теории поля(КТП). Она стала третьим подходом к алгебраическому решению (!!!) и КТП (!!!) решения для нахождения критических свойств некоторой двумерной системы классической статистической физики. Успешными примерами данного подхода являются задача о перколяции и модель Изинга. Второй фактор - это исследование физических и математических процессов планарной геометрии, связанных с практикой разработки новых материалов для микроэлектронных устройств следующего поколения.
	
	\par Что же такое задача о перколяции? В статистической физике существует раздел, называемый "Теория перколяции", которая занимается изучением поведения связных структур на случайных графах. Основным вопросом, которым занимается эта теория заключается в следующем: можно ли пройти от одного конца графа к другому по его элементам. Соответственно разделяют на перколяцию по узлам и перколяцию по связям. В наши дни теория перколяции имеет довольно широкое применение. В качестве примера можно привести ситуации из жизни. Как близко нужно садить деревья в лесу, чтобы, в случае пожара, огонь не сможет перекинуться на близлежащие деревья? Или же когда поливают пористый материал какой-нибудь жидкостью, достигнет ли она дна? В сфере медиа например, определить доступность той или иной машины в сетевом сегменте.
	
	\par Перейдем к модели Изинга. Пусть на двумерной решетке находятся точки, которые имеют определенной состояние, называемое "спин". Он может быть повернут вверх или вниз. Значению "вверх" соответствует +1, а "вниз" -1. Задав каждой точке свое спиновое число, получается конфигурация решетки. У конфигурации есть характеристика, называемая "энергией конфигурации" - она равна сумме произведений всех соседних пар точек,
	\begin{equation}\label{conf_energy}
		E({\sigma}) = - \sum_{(ij)}J_{ij}^{x}\sigma_{i}\sigma_{j} -\sum_{(ik)}J_{ik}^{y}\sigma_{i}\sigma_{k}
	\end{equation}
	
	- где $\sigma$ - конфигурация решетки, $\sigma_{i}$ - точка на решетке, имеющая спин, ($ij$) - сумма по строке и ($ik$) - сумма по столбцу, $J$ - константа связи, являющейся параметром взаимодействия между точками. Статистическая сумма - одна из важных и полезных характеристика в статистической физике. Статсумма представляет собой функцию от температуры, позволяющую получить сведения об энергии, энтропии и других свойствах системы в термодинамическом равновесии. 
	\begin{equation}\label{statsum}
		Z = \sum_{\{\sigma\}}e^{E(\{\sigma\})/T}
	\end{equation}
	- где $T$ - температура, $E(\{\sigma\})$ - энергия конфигурации
	\par Середина XX века примечательна многими событиями, одним из которых является
	развитие теории игр. Основная цель классической теории игр заключается в поиске
	оптимальной стратегии в игре между двумя и более игроками, каждый из которых может
	определять свое поведение. В повторяющейся игре агенты играют несколько раз по 
	тем же правилам. Благодаря этому можно проследить изменения в поведении игрока,
	которое наблюдается из-за того, что он принимает то или иное решение исходя из
	результата предыдущих игр. Иными словами, его стратегия эволюционирует. 
	
	\par Существует множество моделей теории игр, которые являются объектом исследования ученых и по сей день.
	
	\par "Ультиматум" - одна из самых простых моделей теории игр, придумана в 1982г. Между двумя агентами необходимо поделить сумму денег. Один из них предлагает кому сколько достанется, второй же игрок либо соглашается с вариантом и получает свою часть, либо отвергает предложение и деньги никому не достаются. 
	
	\par Модель "Трагедия общинного поля" была придумана ещё в первой половине XIXв Уильямом Фостером Ллойдом. Есть поле, на котором деревенские жители пасут коров.
	Если у всех будет пастись по одной корове, то в этому случае всё будет хорошо, травы на поле хватит, если у кого-то появится вторая корова, то это тоже не критично. Но если каждый житель обзаведется по еще одной корове, тогда на поле не останется травы и весь скот умрет от голода. 
	\par "Проблема вагонетки" состоит в том, готов ли человек толкнуть на рельсы человека, ради спасения вагонетки с пассажирами. Предложена в 1967г Филиппой Фут. 
	
	\par "Аукцион". На продажу выставляется купюра и как и в обычном аукционе происходят торги, в которых участники называют свою цену. Купюру забирает тот, чья ставка была максимальной, но есть один ньюанс - владелец купюры помимо суммы ставки получает также ставку предшествующей максимальной. 
	
	\par Одна из самых распространенных моделей теории игр является Дилемма Заключенного.
	В ней играют два агента, каждый из которых использует две возможные стратегии: 
	кооператор C и дефектор D. Агент получает определенное количество очков от выбора
	своей стратегии и стратегии своего оппонента.
	
	\par У повторяющейся Дилеммы Заключенного также существуют вариации, влияющие на то, как будет эволюционировать поведение агентов, какую стратегию они выберут в той или иной ситуации. Кроме того, модель может иметь большее число стратегий, чем в стандартной интерпретации. Ниже приведены некоторые из них.
	
	\par В одной из вариаций каждый агент со стратегией кооператора всегда будет кооператором на протяжении всей игры и только дефекторы имеют возможность выбора стратегии. Или же ситуация наоборот - когда только кооператоры могут поменять стратегию, а дефекторы остаются дефекторами на протяжении всего времени. 
	
	\par Существует вероятностная модель, в которой вводится некий случайный параметр, отвечающий за то, какую стратегию выберет агент в следующей игре.
	
	\par Также есть модель, в которой агент копирует поведение своего оппонента. Она имеет название "око-за-око". Помимо этого, агент может взять стратегию оппонента, если он дважды выбрал одну и ту же, или наоборот - придерживаться предыдущей стратегии оппонента несколько игр подряд.
	
	\par Существуют и более хитрые вариации. Например, если одна из стратегий принесла агенту хороший доход(после игры с кооператором, если он был кооператором или дефектором), то он будет ее придерживаться, пока она не принесет ему плохой доход. После чего он сменит стратегию и также будет ориентироваться на доход от новой стратегии. 
	
	\par Более того, есть модель, в которой агент придерживается стратегии кооператора, если его оппонент за все предыдущие игры был чаще кооператором, чем дефектором. Здесь приходится определять, что делать, когда оппонент принимал обе стратегии одинаковое количество раз. 
	
	\par Интересно задуманы следующие две модели. В первой агент рассчитывает долю игр оппонента как дефектора и использует ее в качестве вероятности принять данную стратегию самому. Во второй модели агент, прежде чем начинать рассчитывать вышеописанную долю, играет за кооператора некоторое количество игр, затем копирует поведение оппонента, и лишь только потом приступает к вычислению вероятности.
	
	\par Эволюционная теория игр исследует поведение большой популяции, где макроскопическое
	число игроков имеет конечное число стратегий. В то время как классическая теория игр имеет дело с одиночными агентами. Пространственная эволюционная игра заключается в помещении агентов на какую-либо пространственную структуру и во взаимодействии их со своими ближайшими соседями. Структуры могут быть разнообразными - решетки, графы, сети. В качестве модели в текущем исследовании была взята пространственная эволюционная Дилемма Узника на двумерной решетке.
	
	\par Пространственное расположение игроков образует геометрические структуры - группы игроков, которые синхронизируют свое поведение с соседями и конкурируют с другими группами.
	 
	
	\section{Расчетная часть}
	 
	
	\subsection{Предшествующие результаты}
	
	\par Стоит упомянуть, что в этой работе исследование модели Дилеммы Узника началось с варианта Р.Мэя и М.Новака. Ученые расположили агентов на квадратно двумерной решетке. Для минимизации граничных эффектов были взяты периодические граничные условия. Начальная конфигурация составляет $90\%$ кооператоров. Каждый агент играет с восемью ближайшими соседями и с самим собой, получая при этом определенный доход. В следующем раунде агент возьмет стратегию с максимальным доходом среди соседей или же свою(если его стратегия была более прибыльная).
	
	\par В элементарной игре количество очков зависит от выбранной стратегии. Ученые определили следующие правила:
	\begin{itemize}
		\item если оба агента дефекторы, то никто ничего не получает
		\item если оба кооператоры, то оба получают доход $S=1$
		\item при игре разных стратегий дефектор получает максимальный доход $T>S$, а кооператор ничего
	\end{itemize}
	
	\par Ученые пришли к выводу, что данная модель является клеточным автоматом, в котором состояние агентов зависит от окружающих его соседей, которые также зависят от своего окружения. Таким образом, в определении состояния агента в следующем раунде влияет 25 окружающих его агентов(2 уровня соседей и его текущее состояние) и что доход зависит только от одного параметра $b=T/S$. 
	
	\par Кроме того, было изучено поведение единичного агента среди агентов другой стратегии. Так при $b<1$ дефекторы исчезают всегда, а $b>3$ вымирают уже кооператоры. При $1<b<9/5$ возникают описанные учеными некоторые структурные образования, названные глайдерами, ротейторами и гроверы. Малые структурные образования остаются не увеличиваются, а большие становятся меньше. Но в ситуации, когда $b>9/5$ единичный дефектор и любая структура из дефекторов начинает расти, причем единичный дефектор начинает расти как калейдоскоп, что выглядит довольно красиво. Что касается кооператоров, то они растут до того момента, когда $b<2$, и большие структурные образования из кооператоров уменьшаются при $b>2$. Можно заметить, что среди промежутков роста и тех и других агентов, есть промежуток, в котором растут структуры как кооператоров, так и дефекторов, поэтому там и наблюдается некое соперничество между обеими стратегиями.
	
	\par Исследуя пространственную эволюционную Дилемму Узника Р. Мэя и М. Новака, удалось выявить, что конфигурация агентов на решетке резко меняется при переходе через значение $b = 9/5$. При $b<9/5$ структура кооператоров относительно статична, а при $b>9/5$ наоборот можно наблюдать разнообразную динамику на решетке - рост, передвижение и столкновение.
	
	\par Для наглядного представления, как меняется динамика игры с изменением параметра $b$, было взято 25 независимых реализаций начального заполнения и вычислена средняя плотность кооператоров по времени, без учета первых $10^{3}$ игр для уравновешивания и усреднения по $2 x 10^{4}$ раундов. На ГРАФИКЕ можно увидеть результат усреднения, на котором отчетливо виден резкий переход в точке $b=9/5$, при котором плотность кооператоров падает с $~70\% $ до $~30\%$
	
	\par Так же можно заметить одну особенность. Не только плотность изменяется при  переходе через особую точку, также распространение мелких структур резко увеличивается. Но несмотря на это, оно имеет конечный размер, уменьшающий с ростом размера всей решетки. Второй факт заключается в том, что средняя плотность кооператоров на решетке при $b>9/5$ удовлетворяет значению $ f_{C} = 12log2-8 \approx 0.32$.
	
	\subsection{Простая модель со случайным полем}
	
	\par Моделируется пространственная эволюционная игра на основе модели
	Дилеммы Узника, помещенная на квадратную решетку с периодическими граничными
	условиями, в узлах которой находятся игроки одного из двух типов - кооператор 
	или дефектор. Каждый игрок получает доход, играя со своими 8 соседями и со средним полем. Среднее поле определяется следующим образом - вычисляется доля кооператоров на решетке в данный момент времени, генерируется число в интервале от 0 до 1 и сравнивается с долей кооператоров. Если оно оказалось больше, то девятый игрок - дефектор, если меньше - кооператор. В следующем раунде игрок выберет стратегию соседа с максимальным доходом. Случайный игрок окажется кооператором с вероятностью
	равной плотности кооператоров на решетке в данный момент времени. Решетка 
	обновляется, когда каждый игрок определил свой тип в следующем раунде.
	
	\par Ниже в приведены формулы, по которым рассчитываются доходы кооператора ($P_{c}$) и дефектора ($P_{D}$)
	
	$ P_{c}= n_{c}*R+ \theta(f_{c}-x) $
	
	$ P_{D}= n_{c}*R*b+ \theta(f_{c}-x)*b $
	
	где $ f_{c}$ - плотность кооператоров на решетке в данный момент
	времени, $\theta(f_{c}-x)$ - функция Хэвисайда, $x \in [0,1]$- случайное число
	$n_{c}$ - количество соседей, являющихся кооператорами
	 
	
	\par Эта модель практически совпадает с той, что описали в своих работах Р. Мэй и
	М. Новак в 1992г. Отличие составляет лишь то, что в их версии игрок вместо случайного
	игрока играет сам с собой.
	
	\par На \ref{fig:1} представлен пример динамики, происходящей на решетке за 4 поколения.
	Размер решетки 50х50 игроков, параметр $b=1.51$ синим цветом обозначены кооператоры, красным - дефекторы. 
	\begin{figure}[H]
		\centering
		\begin{subfigure}{.5\textwidth}
			\includegraphics[width=1\linewidth]{../conf/grids/pics/2}
			\caption{}
		\end{subfigure}%
		\begin{subfigure}{.5\textwidth}
			\includegraphics[width=1\linewidth]{../conf/grids/pics/3}
			\caption{}
		\end{subfigure}%
	
		\begin{subfigure}{.5\textwidth}
			\includegraphics[width=1\linewidth]{../conf/grids/pics/4}
			\caption{}
		\end{subfigure}%
		\begin{subfigure}{.5\textwidth}
			\includegraphics[width=1\linewidth]{../conf/grids/pics/5}
			\caption{}
		\end{subfigure}%
		\caption{Снимки решеток игры со случайным игроком 50х50 при $b=1.51$}
		\label{fig:1}
	\end{figure}
	
	В приложении находятся графики зависимости плотности кооператоров на решетке 
	от времени при различных значениях параметра $b$. Зеленым цветом показано поведение
	модели Р.Мэй и М.Новака. Синим - модель со случайным игроком. Начальное заполнение решетки 90$\%$ игра длится 21000 поколений. Можно заметить, что сначала присутствует резкое падение вниз плотности кооператоров, а затем с определенного момента изменение 
	стабилизируется, причём для каждого параметра $b$ время релаксации является различным.
	
	\par При подсчете средних плотностей первоначальные изменения плотности не учитываются, поэтому начало подсчета начинается со значений, указанных в \ref{tab1}, где $t_{old}$ - время релаксации в модели Р.Мэй и М.Новак,  $t_{new}$ - в модели со случайным игроком,
	$b$ - максимальный доход в одной игре между двумя игроками.
	
	
	\vspace{10px}
	\begin{center}
	\begin{table}[H]
	\centering
		\begin{tabular}[H]{|c|c|c|}
			\hline 
			b&$t_{old}$& $t_{new} $ \\
			\hline 
			1.3& 100 & 10000 \\ 
			\hline 
			1.35& 100 & 16000 \\ 
			\hline 
			1.4& 100 & 2000 \\ 
			\hline 
			1.45& 100 & 500 \\ 
			\hline 
			1.5& 400 & 100 \\ 
			\hline 
			1.55& 400 & 500 \\ 
			\hline 
			1.6& 16000 & 100 \\ 
			\hline 
			1.65& 100 & 500 \\ 
			\hline 
			1.7& 100 & 4000 \\ 
			\hline 
			1.75& 100 & 1000 \\ 
			\hline 
			1.8& 100 & 500 \\ 
			\hline 
			1.85& 100 & 500 \\ 
			\hline 
			1.9& 100 & 500 \\ 
			\hline 
		\end{tabular}
		\label{tab1}
		\caption{Время релаксации при различных параметра $b$}
	\end{table}
	\end{center}	

	\par Следующий этап - подсчет средней плотности. Для этого запускается 25 реплик игры с количеством поколений выбранным так, чтобы в каждой реплике было 20000 поколений, начиная с времени релаксации, соответствующему параметру $b$. В итоге получается усреднение по 500000 поколениям. 
	
	\begin{figure}
			\centering
			\includegraphics[width=0.7\linewidth]{../compare_av/AVERAGE}
			\caption{График зависимости плотности кооператоров от $b$. Зеленые кружки - модель Р.Мэй и М.Новака, синие треугольники - модель со случайным игроком}
			\label{fig:average}
	\end{figure}
	
	В Приложение 3 находится таблица значений величин изображенных на \ref{average}. На графике можно заметить, что практически все точки, в которых происходит скачек средней плотности совпадают в обеих моделях. Но в модели со случайным игроком есть точка перехода $b=1.66(6)$, после которой кооператоры вымирают полностью, не совпадет с другой моделью. Аналогичной точкой в модели Р.Мэй и М.Новака является $b=3$
	
	\par Кроме того, ученые выяснили, что в точке $b=9/5$ начинает расти единичный дефектор, причем образуя при этом эволюционирующий калейдоскоп. Аналогичной точкой в модели со случайным игроком является $b=3/2$. Слева от этой точки в модели Р.Мэй и М.Новака единичный дефектор живет следующим образом: 1D-9D-1D (чередование - 1 дефектор, 9 дефекторов, и снова1 дефектор). В модели со случайным игроком происходит немного другой процесс: 1D-9D-...-8D-6D. Сначала один дефектор разрастается на 9(квадрат 3х3), затем спустя некоторое время у него пропадает один уголок, далее через некоторое число поколений у него остается 6.(см Приложение 3).
	
	\par При параметре $b>3/2$ Происходит рост единичного дефектора, причём аналогичным образом, как и в другой модели при $b=9/5$, образуется эволюционирующий калейдоскоп, но так как в исследуемой модели присутствует фактор случайности, то фрактал рушится спустя несколько поколений после начала. (Приложение 4)

	\section{Приложение 1}
	\begin{figure}[H]
		\begin{subfigure}{.5\textwidth}
		\includegraphics[width=.8\linewidth]{../compare/1.3_full}
		\caption{1.3}
		\label{fig:21}
		\end{subfigure}
		\begin{subfigure}{.5\textwidth}
		\includegraphics[width=.8\linewidth]{../compare/1.35_fulll}
		\caption{1.35}
		\label{fig:22}
		\end{subfigure}%
	
		\begin{subfigure}{.5\textwidth}
		\includegraphics[width=.8\linewidth]{../compare/1.4_old_slice}
		\caption{1.4}
		\label{fig:23}
		\end{subfigure}
		\begin{subfigure}{.5\textwidth}
		\includegraphics[width=.8\linewidth]{../compare/1.45_old_slice}
		\caption{1.45}
		\label{fig:24}
		\end{subfigure}%
	
		\begin{subfigure}{.5\textwidth}
		\includegraphics[width=.8\linewidth]{../compare/1.5_old_slice}
		\caption{1.5}
		\label{fig:25}
		\end{subfigure}
		\begin{subfigure}{.5\textwidth}
		\includegraphics[width=.8\linewidth]{../compare/1.55_old_slice}
		\caption{1.55}
		\label{fig:26}
		\end{subfigure}%
	
		\begin{subfigure}{.5\textwidth}
		\includegraphics[width=.8\linewidth]{../compare/1.6_full}
		\caption{1.6}
		\label{fig:27}
		\end{subfigure}
		\begin{subfigure}{.5\textwidth}
		\includegraphics[width=.8\linewidth]{../compare/1.65_old_slice}
		\caption{1.65}
		\label{fig:28}
		\end{subfigure}%
	\end{figure}
	\begin{figure}
		\ContinuedFloat
		\begin{subfigure}{.5\textwidth}
		\includegraphics[width=.8\linewidth]{../compare/1.7_old_slice}
		\caption{1.7}
		\label{fig:29}
		\end{subfigure}
		\begin{subfigure}{.5\textwidth}
		\includegraphics[width=.8\linewidth]{../compare/1.75_old_slice}
		\caption{1.75}
		\label{fig:30}
		\end{subfigure}%
	
	
		\begin{subfigure}{.5\textwidth}
		\includegraphics[width=.8\linewidth]{../compare/1.8_old_slice}
		\caption{1.8}
		\label{fig:31}
		\end{subfigure}
		\begin{subfigure}{.5\textwidth}
		\includegraphics[width=.8\linewidth]{../compare/1.85_old_slice}
		\caption{1.85}
		\label{fig:32}
		\end{subfigure}%
	
	\label{fig:2}
	\caption{Графики}
	\end{figure}	

 \section{Приложение 2}
 
 	\begin{center}
 		\begin{table}[H]
 			\centering
 			\begin{tabular}[H]{|c|c|c|c|c|}
 				\hline 
 				b&$f^{Prob}_{c}$& $err_{Prob}$&$f^{MN}_{c}$& $err_{MN}$ \\
 				\hline
 				1.3&0.9931174116&0.00185784485341&0.8570233712&0.0143276179104 \\ 
 				\hline 
 				1.35&0.9754565452&0.00346478875112&0.856210325&0.0125231036421 \\ 
 				\hline 
 				1.39&0.9754565452&0.00346478875112&0.8830397906&0.00693414394232 \\ 
 				\hline 
 				1.45&0.6597172636&0.0108623201089&0.8491049608&0.00729272717975 \\ 
 				\hline 
 				1.5&0.509002332&0.0113359535395&0.8663578614&0.0113117064172 \\ 
 				\hline 
 				1.51&0.4429103238&0.0119655849564&0.8123875454&0.0140520859428 \\ 
 				\hline 
 				1.55&0.4429103238&0.0119655849564&0.8123875454&0.0140520859428 \\ 
 				\hline 
 				1.59&0.4429103238&0.0119655849564&0.8123875454&0.0140520859428 \\ 
 				\hline 
 				1.6&0.2961826392&0.0149808933598&0.7138630518&0.0181247181071 \\ 
 				\hline 
 				1.65&0.180777846&0.0153771078916&0.702671413&0.0107453568321 \\ 
 				\hline 
 				1.6666&0.180777846&0.0153771078916&& \\ 
 				\hline 
 				1.6667&0.0002228936&0.00138441836027&& \\ 
 				\hline 
 				1.7&2.4186e-06&0.000111590838689&0.7125603104&0.00846476171667 \\ 
 				\hline 
 				1.75&4.39404e-05&0.0004237380327&0.7550077526&0.00784245956098 \\ 
 				\hline 
 				1.8&6.2721e-05&0.000427369981922&0.4079621234&0.0151591363943 \\ 
 				\hline 
 				1.85&4.90412e-05&0.00039349382508&0.3156021008&0.01742878026 \\ 
 				\hline 
 				1.9&4.90412e-05&0.00039349382508&0.3156021008&0.01742878026 \\
 				\hline		
 			\end{tabular}
 			\label{tab1}
 			\caption{Средняя плотность кооператоров моделей со случайным игроком(Prob) и моделью Р.Мэй и М.Новак (MN) при различных параметра $b$, err - ошибка}
 		\end{table}
 	\end{center}
 
 	\section{Приложение 3}	
 	\begin{figure}[H]
	
	\begin{subfigure}{.5\textwidth}
		\caption{}
		\label{fig:n1351}
		\includegraphics[width=.8\linewidth]{../conf/defs/pics/n1351}
	\end{subfigure}%
	\begin{subfigure}{.5\textwidth}
		\caption{}
		\label{fig:o1351}
		\includegraphics[width=.8\linewidth]{../conf/defs/pics/o1351}
	\end{subfigure}%

	\begin{subfigure}{.5\textwidth}
		\caption{}
		\label{fig:n1351}
		\includegraphics[width=.8\linewidth]{../conf/defs/pics/n1352}
	\end{subfigure}%
	\begin{subfigure}{.5\textwidth}
		\caption{}
		\label{fig:n1351}
		\includegraphics[width=.8\linewidth]{../conf/defs/pics/o1352}
	\end{subfigure}%

	\begin{subfigure}{.5\textwidth}
		\caption{}
		\label{fig:n1351}
		\includegraphics[width=.8\linewidth]{../conf/defs/pics/n1353}
	\end{subfigure}%
	\begin{subfigure}{.5\textwidth}
		\caption{}
		\label{fig:n1351}
		\includegraphics[width=.8\linewidth]{../conf/defs/pics/o1351}
	\end{subfigure}%

	\begin{subfigure}{.5\textwidth}
		\caption{}
		\label{fig:n1351}
		\includegraphics[width=.8\linewidth]{../conf/defs/pics/n1354}
	\end{subfigure}%
	\begin{subfigure}{.5\textwidth}
		\caption{}
		\label{fig:n1351}
		\includegraphics[width=.8\linewidth]{../conf/defs/pics/o1352}
	\end{subfigure}%
	\caption{Развитие единичного дефектора при $b<1.5$}
	\end{figure}

\section{Приложение 4}	
\begin{figure}[H]
	
	\begin{subfigure}{.5\textwidth}
		\caption{}
		\label{fig:n1351}
		\includegraphics[width=.8\linewidth]{../conf/defs/pics/n1551}
	\end{subfigure}%
	\begin{subfigure}{.5\textwidth}
		\caption{}
		\label{fig:o1351}
		\includegraphics[width=.8\linewidth]{../conf/defs/pics/o1811}
	\end{subfigure}%
	
	\begin{subfigure}{.5\textwidth}
		\caption{}
		\label{fig:n1351}
		\includegraphics[width=.8\linewidth]{../conf/defs/pics/n1552}
	\end{subfigure}%
	\begin{subfigure}{.5\textwidth}
		\caption{}
		\label{fig:n1351}
		\includegraphics[width=.8\linewidth]{../conf/defs/pics/o1812}
	\end{subfigure}%
	
	\begin{subfigure}{.5\textwidth}
		\caption{}
		\label{fig:n1351}
		\includegraphics[width=.8\linewidth]{../conf/defs/pics/n1553}
	\end{subfigure}%
	\begin{subfigure}{.5\textwidth}
		\caption{}
		\label{fig:n1351}
		\includegraphics[width=.8\linewidth]{../conf/defs/pics/o1813}
	\end{subfigure}%
	
	\begin{subfigure}{.5\textwidth}
		\caption{}
		\label{fig:n1351}
		\includegraphics[width=.8\linewidth]{../conf/defs/pics/n1554}
	\end{subfigure}%
	\begin{subfigure}{.5\textwidth}
		\caption{}
		\label{fig:n1351}
		\includegraphics[width=.8\linewidth]{../conf/defs/pics/o1814}
	\end{subfigure}%

	\caption{Развитие единичного дефектора при $b>1.5$}
	
\end{figure}
\begin{figure}
	\ContinuedFloat
		\begin{subfigure}{.5\textwidth}
		\caption{}
		\label{fig:n1351}
		\includegraphics[width=.8\linewidth]{../conf/defs/pics/n1555}
	\end{subfigure}%
	\begin{subfigure}{.5\textwidth}
		\caption{}
		\label{fig:n1351}
		\includegraphics[width=.8\linewidth]{../conf/defs/pics/o1815}
	\end{subfigure}%
	
	
	\begin{subfigure}{.5\textwidth}
		\caption{}
		\label{fig:n1351}
		\includegraphics[width=.8\linewidth]{../conf/defs/pics/n1556}
	\end{subfigure}%
	\begin{subfigure}{.5\textwidth}
		\caption{}
		\label{fig:n1351}
		\includegraphics[width=.8\linewidth]{../conf/defs/pics/o1816}
	\end{subfigure}%

	\begin{subfigure}{.5\textwidth}
		\caption{}
		\label{fig:n1351}
		\includegraphics[width=.8\linewidth]{../conf/defs/pics/n1557}
	\end{subfigure}%
	\begin{subfigure}{.5\textwidth}
		\caption{}
		\label{fig:n1351}
		\includegraphics[width=.8\linewidth]{../conf/defs/pics/o1817}
	\end{subfigure}%

	\caption{Развитие единичного дефектора при $b>1.5$}
\end{figure}
\end{document}